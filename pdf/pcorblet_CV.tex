%%%%%%%%%%%%%%%%%%%%%%%%%%%%%%%%%%%%%%%%%
% Medium Length Professional CV
% LaTeX Template
% Version 2.0 (8/5/13)
%
% This template has been downloaded from:
% http://www.LaTeXTemplates.com
%
% Original author:
% Trey Hunner (http://www.treyhunner.com/)
%
% Important note:
% This template requires the resume.cls file to be in the same directory as the
% .tex file. The resume.cls file provides the resume style used for structuring the
% document.
%
%%%%%%%%%%%%%%%%%%%%%%%%%%%%%%%%%%%%%%%%%

%----------------------------------------------------------------------------------------
%	PACKAGES AND OTHER DOCUMENT CONFIGURATIONS
%----------------------------------------------------------------------------------------

\documentclass{resume} % Use the custom resume.cls style

\usepackage[left=0.75in,top=0.1in,right=0.75in,bottom=0.6in]{geometry} % Document margins
\usepackage{hyperref}
\newcommand{\tab}[1]{\hspace{.2667\textwidth}\rlap{#1}}
\newcommand{\itab}[1]{\hspace{0em}\rlap{#1}}
% \name{Pauline Corblet} % Your name
% \address{Sciences Po Economics Department} % Your address
% %\address{123 Pleasant Lane \\ City, State 12345} % Your secondary addess (optional)
% \address{pauline.corblet@sciencepo.fr} % Your phone number and email

\begin{document}

{\bf \large PAULINE CORBLET} \hfill pauline.corblet@sciencespo.fr\\
{\phantom{smth}} \hfill \href{https://paulinecorblet.github.io}{paulinecorblet.github.io} \\
New York University, Saadiyat Island, \\ 
Abu Dhabi, UAE \\

{\large Research interests: matching theory, labor economics, education, structural econometrics}

%----------------------------------------------------------------------------------------
%	ACADEMIC EMPLOYMENT SECTION
%----------------------------------------------------------------------------------------

\begin{rSection}{Academic Employment}

{\bf New York University Abu Dhabi} \hfill {2023-} \\
  Assistant Professor - Tenure Track

{\bf University of Luxembourg} \hfill {2022-2023} \\
Postdoctoral Researcher

\end{rSection}
%----------------------------------------------------------------------------------------
%	EDUCATION SECTION
%----------------------------------------------------------------------------------------

\begin{rSection}{Education}

{\bf Sciences Po} \hfill {2018-2022}
\\ Ph.D. in Economics

{\bf Sciences Po} \hfill {2015-2017}
\\ M.Res. in Economics, with honors

{\bf Sorbonne Université} \hfill {2014-2017}
\\ M.Res in Applied Mathematics, with honors

{\bf Sciences Po-Sorbonne Université} \hfill {2011-2014}
\\ Dual Bachelor degree in mathematics and social sciences, with honors


\end{rSection}
% %----------------------------------------------------------------------------------------
% %	TECHNICAL STRENGTHS SECTION
% %----------------------------------------------------------------------------------------
%
% \begin{rSection}{Technical Strengths}
%
% \begin{tabular}{ @{} >{\bfseries}l @{\hspace{6ex}} l }
% Computer Languages &  C/C++, MATLAB \\
% Software \& Tools & HTML, LaTeX, Excel, Gerris, Mathematica, ASPEN Plus, Tecplot \\
% \end{tabular}
%
% \end{rSection}

%----------------------------------------------------------------------------------------
%	TEACHING SECTION
%----------------------------------------------------------------------------------------

\begin{rSection}{Teaching}

{\bf Data Analysis, NYUAD} \hfill {Fall 2023 \& 2024}
  \\ Lecturer

{\bf Graduate Econometrics 2, Sciences Po} \hfill {Spring 2021}
  \\ Teaching assistant

{\bf Graduate Macroeconomics 1, Sciences Po} \hfill {Fall 2020}
  \\ Teaching assistant

% {\bf Math$+$Econ$+$Code masterclass, New York University, Paris campus} \hfill {\em June 2019}
% \\ Teaching assistant

{\bf Intermediate Microeconomics, Sciences Po} \hfill {Fall 2018 \& 2019}
\\ Course coordinator and undergraduate students instructor

\end{rSection}

%----------------------------------------------------------------------------------------
%	WORK EXPERIENCE SECTION
%----------------------------------------------------------------------------------------

\begin{rSection}{Previous Work experience}

  % {\bf Graduate Survey coordinator, Sciences Po Careers} \hfill {2018-2019} \\
  % With Cyriel Pelletier, Anne Lesegretain

{\bf Analyst, Compass Lexecon} \hfill {2017-2018}
\\ Anti-trust and Energy departments

% {\bf Research assistant, Sciences Po} \hfill {2016}
% \\ With Prof. Alfred Galichon

\end{rSection}

%----------------------------------------------------------------------------------------
%	PROJECTS EXPERIENCE SECTION
%----------------------------------------------------------------------------------------

\begin{rSection}{Working Papers}

  {\bf Education expansion, Sorting, and the Decreasing Education Wage Premium}, Job market paper. ‘Honorable Mention' Award at the 2022 IAAE Conference.
  
  {\bf The Decreasing Resturns to Experience for Higher Education Graduates}

  {\bf Late Bloomers? The Causes and Consequences of Getting Education Later in Life}, with Zsófia Bárány and Moshe Buchinsky

  {\bf Modeling Monospony on the Labor Market with Separable Matching Models}, with Arnaud Dupuy
  
  \end{rSection}

\begin{rSection}{Ongoing Projects}

{\bf Repeated Matching Games: an Empirical Framework}, with Jeremy Fox and Alfred Galichon

\end{rSection}

\begin{rSection}{Packages}

{\bf NonLinearProg}, a Julia wrapper to call non linear optimization routines, developed with Simon Weber
\end{rSection}

%----------------------------------------------------------------------------------------
%	PRESENTATIONS SECTION
%----------------------------------------------------------------------------------------

\begin{rSection}{Seminars and Conferences}
\begin{itemize}
  \item 2025: New Perspetives for Policy Evaluation, NYUAD
  \item 2024: NYU Econometrics seminar, NYUAD Global Dynamics seminar, Cargese Optimal Transport Workshop, Louvain IRES seminar, Exeter Econometrics seminar, Bristol econometrics seminar, Rice Structural Metrics Conference.  
  \item 2023: IAAE 2023, DEM internal seminar, AFSE 71th congress, Leuven Summer Event, University of York, CREST, DG EMPL Triple A Talks.
  \item 2022: University of Vienna AES seminar, IAAE 2022, AFSE 70th Congress, Hi! Paris Symposium, NYUAD Postdoc workshop, Duke University, Toulouse School of Economics, Uppsala University, Stockholm University.
  \item 2021: New York University Abu Dhabi, Dale T. Mortensen Conference, Paris workshop on Optimal Transports with applications to Economics and Statistics, Ecomod conference , Equiprice Seminar.
  \item 2020: Sciences Po Lunch Seminar.
  \item 2019: CEREQ Workshop ‘‘Enquêtes Générations'', Sorbonne Université LPSM external seminar, Sciences Po PhD internal seminar.
\end{itemize}

{\em Organizer} - PhD internal seminar (Sciences Po) \hfill {2019-2021}

\end{rSection}


\end{document}
